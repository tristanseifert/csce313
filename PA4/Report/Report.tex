\documentclass[10pt]{article}
\usepackage[margin=.7in]{geometry}
\usepackage[T1]{fontenc}
%\usepackage{amsmath,amssymb,amsthm}
%\usepackage{enumerate}
%\usepackage{minibox}
%\usepackage[table]{xcolor}

\usepackage{csvsimple}
\usepackage{booktabs}
\usepackage{tikz}
\usepackage{pgfplots}

\usepackage{underscore}

\pgfplotsset{width=12cm,compat=1.3}

% \def\code#1{\texttt{#1}}

%%%%%%%%%%%%%%%%%%%%%%%%%%%%%%%%%%%%%%%%%%%%%%%%%%%%%%%%%%%%%%%%%%%%%%%%%
\title{CSCE 313.506-F18: \\Programming Assignment 4}
\author{Tristan Seifert}
\date{Due: Thursday, Nov. 1, 2018}

%%%%%%%%%%%%%%%%%%%%%%%%%%%%%%%%%%%%%%%%%%%%%%%%%%%%%%%%%%%%%%%%%%%%%%%%%
\begin{document}
\maketitle

\hrule

%%%%%%%%%%%%%%%%%%%%%%%%%%%%%%%%%%%%%%%%%%%%%%%%%%%%%%%%%%%%%%%%%%%%%%%%%
\section{Performance Data}
This data was gathered for varying values of $w$ while $n = 10000$. The system used runs macOS 10.14.1 beta 2, with 64GB of RAM and dual Xeon E5-2670 v2 (10 physical cores, 20 virtual cores) processors at 2.6 GHz.

A maximum of approximately 2000 threads are possible until the file descriptor limit of 4096 was reached. If the program reaches this, calls to create a new channel return with \texttt{EMFILE}; the program prints the message \texttt{Too many open files} and attempts to continue with the threads that have already been spawned.

\begin{figure}[h]
\centering
\begin{tikzpicture}[scale=1, transform shape]
\pgfplotsset{
    scale only axis,
}

	\begin{axis}[
  axis y line*=left,
		xlabel={Number of Threads ($w$)},
		ylabel=Time (seconds),
		xmode=log,
		%log basis y=10
	]
	\addplot[color=red,smooth,mark=o] coordinates {
(5,1.87483)
(10,1.69082)
(15,1.55342)
(25,1.40466)
(35,1.20532)
(45,1.13468)
(55,1.13611)
(65,1.06531)
(75,1.03595)
(85,1.01859)
(95,0.99349)
(105,0.96444)
(115,0.98367)
(125,0.97590)
(135,0.95225)
(145,0.93276)
(155,0.96534)
(165,0.97023)
(175,0.96852)
(185,0.96476)
(200,0.87838)
(250,2.50741)
(300,1.35759)
(400,4.53737)
(500,11.30150)
(600,8.06313)
(700,7.74134)
(800,33.90010)
(900,23.63470)
(1000,95.96800)
(1500,411.91800)
(2000,437.50200)
	};
	\addplot[color=blue,smooth,mark=o] coordinates {
(5,1.82058)
(10,1.58102)
(15,1.45176)
(25,1.56471)
(35,1.28997)
(45,1.16334)
(55,1.12449)
(65,1.05702)
(75,1.07653)
(85,1.02945)
(95,0.997434)
(105,0.970505)
(115,0.964283)
(125,0.964424)
(135,1.02618)
(145,0.919301)
(155,0.925374)
(165,0.902376)
(175,0.914494)
(185,0.937188)
(200,0.8306)
(250,0.86805)
(300,0.916199)
(400,0.96884)
(500,1.04914)
(600,1.26772)
(700,1.09808)
(800,9.24121)
(900,21.8044)
(1000,2.46151)
(1500,49.6417)
(2000,432.654)
	};
	\end{axis}
	
		\matrix [draw,below left] at (current bounding box.north west) {
  \node [style={draw = red,shape=circle},label=right:{b = 100}] {}; \\
  \node [style={draw = blue,shape=circle},label=right:{b = 1000}] {}; \\
};
	
\end{tikzpicture}
\caption{Runtimes for $n = 10000$}
\end{figure}

Compared to PA3, this program does run a fair bit slower, which can be attributed to the additional synchronization and error checking in the \texttt{BoundedBuffer} class.

Unlike PA3, the optimal number of worker threads appears to be around 105-115, regardless of the buffer size. This is roughly 5x more than PA3, but can be explained because the threads will be waiting for data to arrive on the buffer instead of all running simultaneously. Better performance can possibly be attained by tweaking the size of the request buffer size in comparison to the total number of requests and worker threads.

\begin{table}[h]
\begin{tabular}{c|c|c}
\textbf{Worker Threads} & \textbf{b = 100} & \textbf{b = 1000} \\ \hline
5 & 1.87483 & 1.82058 \\
10 & 1.69082 & 1.58102 \\
15 & 1.55342 & 1.45176 \\
25 & 1.40466 & 1.56471 \\
35 & 1.20532 & 1.28997 \\
45 & 1.13468 & 1.16334 \\
55 & 1.13611 & 1.12449 \\
65 & 1.06531 & 1.05702 \\
75 & 1.03595 & 1.07653 \\
85 & 1.01859 & 1.02945 \\
95 & 0.99349 & 0.997434 \\
105 & 0.96444 & 0.970505 \\
115 & 0.98367 & 0.964283 \\
125 & 0.97590 & 0.964424 \\
135 & 0.95225 & 1.02618 \\
145 & 0.93276 & 0.919301 \\
155 & 0.96534 & 0.925374 \\
165 & 0.97023 & 0.902376 \\
175 & 0.96852 & 0.914494 \\
185 & 0.96476 & 0.937188 \\
200 & 0.87838 & 0.8306 \\
250 & 2.50741 & 0.86805 \\
300 & 1.35759 & 0.916199 \\
400 & 4.53737 & 0.96884 \\
500 & 11.30150 & 1.04914 \\
600 & 8.06313 & 1.26772 \\
700 & 7.74134 & 1.09808 \\
800 & 33.90010 & 9.24121 \\
900 & 23.63470 & 21.8044 \\
1000 & 95.96800 & 2.46151 \\
1500 & 411.91800 & 49.6417 \\
2000 & 437.50200 & 432.654
\end{tabular}
\end{table}


\end{document}
