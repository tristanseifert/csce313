\documentclass[10pt]{article}
\usepackage[margin=.7in]{geometry}
\usepackage[T1]{fontenc}

\usepackage{xcolor}
\usepackage{minted}


% \def\code#1{\texttt{#1}}

%%%%%%%%%%%%%%%%%%%%%%%%%%%%%%%%%%%%%%%%%%%%%%%%%%%%%%%%%%%%%%%%%%%%%%%%%
\title{CSCE 313.506-F18: \\Programming Assignment 2}
\author{Tristan Seifert}
\date{Due: Sunday, Oct. 7, 2018}

%%%%%%%%%%%%%%%%%%%%%%%%%%%%%%%%%%%%%%%%%%%%%%%%%%%%%%%%%%%%%%%%%%%%%%%%%
\begin{document}
\maketitle

\hrule

%%%%%%%%%%%%%%%%%%%%%%%%%%%%%%%%%%%%%%%%%%%%%%%%%%%%%%%%%%%%%%%%%%%%%%%%%
\section{Parsing}
User input is parsed in four stages – first, by splitting the string by pipes; then, parsing input/output redirection, followed by finding an ampersand to put processes in background, and lastly, splitting the process arguments by spaces. During each of these stages, quoted strings are treated as a single token -- so a pipe or an ampersand in a quoted string will be ignored, for example.

At the end of parsing, the user's inputted command line is transformed into one or more \texttt{Fragment} classes, which are then executed by the \texttt{Shell} class.

\texttt{Shell} checks if the program name is one of the built-ins -- \texttt{cd, exit, jobs} -- and if it is, simply runs a function to handle it. If the program name cannot be executed without launching another process, \texttt{execvp()} is used to launch it, and any other processes.


\section{Piping}
When the shell determines that a single command line resulted in two or more \texttt{Fragment}s, it sets up $n$ number of pipes, where $n$ is equal to the number of fragments.

\begin{minted}{c}
// Create pipes (one per fragment; [0] is stdin, [1] is stdout)
int pipes[fragments.size()][2];

for(int i = 0; i < fragments.size(); i++) {
  err = pipe(pipes[i]);
}
\end{minted}

The standard output of the first fragment connects to the input of the second, and so forth, until all fragments have had their \texttt{stdin}/\texttt{stdout} redirected. If needed, the first process' input may be read from file, while the last one's output may be written to a file.

\begin{minted}{c}
// connect stdin
int stdinIndex = (i - 1);

if(stdinIndex >= 0) {
  err = dup2(pipes[stdinIndex][0], STDIN_FILENO);
}

// connect stdout for all but the last process
int stdOutIndex = (i - 0);

if(i < (fragments.size() - 1)) {
  err = dup2(pipes[stdOutIndex][1], STDOUT_FILENO);
}
\end{minted}

Note that while the pipes are created in the shell's process, connecting them to the \texttt{stdin}/\texttt{stdout} takes place in the child process: e.g. after \texttt{fork()} returns zero; additionally, error handling was omitted in this excerpt.

\section{Redirection}
When IO redirection is desired, the specified files are opened for reading or writing, and replace the existing standard input or output using the \texttt{dup2()} syscall:

\begin{minted}{c}
// redirecting standard input
int newStdin = open(it->stdinFile.c_str(), O_RDONLY);
dup2(newStdin, STDIN_FILENO);

// redirecting standard output
int newStdout = open(it->stdoutFile.c_str(), (O_RDWR | O_TRUNC | O_CREAT), 0644);
dup2(newStdout, STDOUT_FILENO);
\end{minted}

Note that this redirection takes place in the child process: e.g. after \texttt{fork()} returns zero; additionally, error handling was omitted in this excerpt.

When executing a single process, both \texttt{stdin} and \texttt{stdout} can be redirected; but when pipes are involved, the first process can only have its \texttt{stdin} redirected, the last process can only have its \texttt{stdout} redirected, while any other process between the first and last cannot have file redirection.

\section{Customization}
The shell can be customized with a \texttt{.kushrc} file in the directory from which it is executed. This file is structured like an INI file, and might look something like this:

\begin{minted}{ini}
[prompt]
# Text to show as the prompt. You can substitute $STATUS, $USER, $HOST, $DIR
# $DATE and $TIME. A trailing space is automatically added.
# To colorize the status code (0 is green, any other color is red) you can
# use $COLORSTATUS.
text = {$COLORSTATUS} \e[31m$USER\e[0m@\e[34m$HOST\e[0m - \e[32m$DIR\e[0m ($DATE $TIME) $
\end{minted}

In this case, the ANSI escape sequences (\texttt{\textbackslash e[31m} and so forth) are used to colorize the prompt. This example prompt might look like the following:

\colorbox{black}{
\color{white}
\tt

\{{\color{green}0}\} {\color{red}tristan}@{\color{cyan}Bird-Machine.local} - {\color{green}/Users/tristan/TamuCS/CSCE313/PA2} (10/07/18 19:21:34) \$
}

The shell will substitute \texttt{\$COLORSTATUS} with a colorized version of the \texttt{\$STATUS} variable -- green if the status code is zero, red otherwise.

\end{document}
